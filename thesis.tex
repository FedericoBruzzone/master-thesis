\documentclass{adapt-lab}

\usepackage{pifont}
\usepackage{rotating}

\newcommand{\cwcomment}[1]{\profcomment[Walter]{#1}}
\newcommand{\bfcomment}[1]{\studcomment[Federico]{#1}}

%\usepackage{amsthm}
%\usepackage{amssymb}
%\usepackage{amsmath}
%\usepackage{subfig}

% ========== My settings ==========
\usetikzlibrary{angles,shadows.blur,positioning,backgrounds}
\usepackage{forest}

\usepackage{bussproofs} % for proof trees
\usepackage{booktabs} % for tables
\usepackage{amsmath} % for math

\newcommand{\circleblack}{\CIRCLE}
\newcommand{\circlewhite}{\Circle}
\newcommand{\circleleft}{\LEFTcircle}
\newcommand{\circleright}{\RIGHTcircle}

\tcbset{
    myboxstyle/.style={
        enhanced, breakable,
        sharp corners,
        % left skip=8mm,
        % attach boxed title to top left={yshift*=-\tcboxedtitleheight/2},
        boxed title style={colframe=#1!40, height=6mm, bean arc, boxrule=1pt},
        colback=#1!10, colframe=#1!30, boxrule=1pt,
        coltitle=black, colbacktitle=#1!10,
        fonttitle=\bfseries,
    }
}

% \fboxsep=10mm %padding thickness
\fboxrule=1.5pt %border thickness

\hypersetup{
    colorlinks = true
}

\tcbuselibrary{theorems}
\newtcbtheorem{mydefinition}{Definition}%
{myboxstyle=yellow}{def}
% \newtcbtheorem[number within=section]{mytheo}{My Theorem}%
% {colback=green!5,colframe=green!35!black,fonttitle=\bfseries}{th}

\RequirePackage{xspace}
\RequirePackage[tikz]{bclogo}
\definecolor{bgblue}{RGB}{245,243,253}
\definecolor{ttblue}{RGB}{91,194,224}
\renewcommand\bcStyleTitre[1]{\large\textcolor{ttblue}{#1}}
\newenvironment{Caveats}{\begin{bclogo}[couleur=bgblue, arrondi =0 , logo=\bcbombe, barre=none,noborder=true]{Caveats}}{\end{bclogo}}
\newenvironment{Tips}{\begin{bclogo}[couleur=bgblue, arrondi =0 , logo=\bclampe, barre=none,noborder=true]{Tips \& Tricks}}{\end{bclogo}}
\newenvironment{Info}{\begin{bclogo}[couleur=bgblue, arrondi =0 , logo=\bcinfo, barre=none,noborder=true]{Info}}{\end{bclogo}}
\newenvironment{Warning}{\begin{bclogo}[couleur=bgblue, arrondi =0 , logo=\bcattention, barre=none,noborder=true]{Warning}}{\end{bclogo}}


\newtheorem{definition}{Formal Definition}
% =================================

\begin{document}

% \title{Code Less to Code More}
% \subtitle{Streamlining LSP Development for \\ Language Families \vspace{2cm}}
% \title{Moving from $\mathcal{L} + \mathcal{E}$ to $\mathcal{L} \times 1$ combinations to support LSP for $\mathcal{L}$ languages}
% \subtitle{\vspace{3pt}\small Moving from $\mathcal{L} + \mathcal{E}$ to $\mathcal{L} \times 1$ combinations to support LSP for $\mathcal{L}$ languages}
\title{Toward a modular approach for type systems and LSP \\ generation \\}
% \subtitle{Moving from $\mathcal{L} + \mathcal{E}$ to $\mathcal{L} \times 1$ combinations to support LSP for $\mathcal{L}$ languages \\}
\author{Federico Cristiano Bruzzone}
\matricola{27427A}
\ayear{2023-2024}
\advisor{Prof.\,Walter Cazzola}
\coadvisor{Dr.\,Luca Favalli}
% \school{Universit\`a degli Studi di Milano}
\degree{MSc in Computer Science}
\selectlanguage{english}

\makefirstpage

\frontmatter

\tableofcontents

\mainmatter

% \input chapters/ex-introduction.tex
% \input chapters/ex-newchap.tex

\input chapters/introduction.tex
\input chapters/background.tex
\input chapters/related-work.tex
\input chapters/concept.tex
\input chapters/implementation.tex
\input chapters/evaluation.tex
\input chapters/conclusions.tex
\input chapters/tests.tex

% \inlinerust{fn}

\appendix

\bibliographystyle{plain}
\bibliography{local,strings,reflection,aosd,my_work,oolanguages,programming,software_engineering,dsl,pl,distributed_systems,grammars,pattern,biomedicine,splc,roles}

\end{document}
