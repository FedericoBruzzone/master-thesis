\chapter{Tests}

\begin{tcolorbox}%[breakable, enhanced]
   \small\textbf{
       Test
   }\\~\\
   This is a test.
   \begin{compactitem}
      \item \textit{Test1}---This is a test.
      \item \textit{Test2}---This is another test.
      \item \textit{Test3}---This is yet another test.
   \end{compactitem}
\end{tcolorbox}

% Error
\begin{tcolorbox}[colback=red!5!white,colframe=red!75!black,title=My Heading]
This is a \textbf{tcolorbox}.
\tcblower
Here, you see the lower part of the box.
\end{tcolorbox}

% Warning
\begin{tcolorbox}[colback=yellow!5!white,colframe=yellow!75!black,title=My Heading]
This is a \textbf{tcolorbox}.
\tcblower
Here, you see the lower part of the box.
\end{tcolorbox}

% Info
\begin{tcolorbox}[colback=blue!5!white,colframe=blue!75!black,title=My Heading]
This is a \textbf{tcolorbox}.
\tcblower
Here, you see the lower part of the box.
\end{tcolorbox}

% Success
\begin{tcolorbox}[colback=green!5!white,colframe=green!75!black,title=My Heading]
This is a \textbf{tcolorbox}.
\tcblower
Here, you see the lower part of the box.
\end{tcolorbox}

\newcounter{myboxcounter}
\tcbset{
    myboxstyle/.style={
        enhanced, breakable,
        sharp corners,
        left skip=8mm,
        attach boxed title to top left={yshift*=-\tcboxedtitleheight/2},
        boxed title style={colframe=#1!40, height=6mm, bean arc, boxrule=1pt},
        colback=#1!10, colframe=#1!30, boxrule=1pt,
        coltitle=black, colbacktitle=#1!10,
        fonttitle=\bfseries,
        underlay boxed title={%
            \node[circle, fill=#1!10, draw=#1!40, inner sep=1pt, minimum size=6mm, line width=1pt, font=\bfseries] (boxnumber) at ([xshift=-6mm]title.west) {\thetcbcounter};
        },
        underlay unbroken={%
            \draw[#1!40, line width=1pt] (title.west)--(boxnumber);
            \draw[#1!40, -{stealth}, line width=1pt] (boxnumber)--(boxnumber|-frame.south);
        },
        underlay first={%
            \draw[#1!40, line width=1pt] (title.west)--(boxnumber);
            \draw[#1!40, line width=1pt] (boxnumber)--(boxnumber|-frame.south);
        },
        underlay middle={%
            \draw[#1!40, line width=1pt] (boxnumber|-frame.north)--(boxnumber|-frame.south);
        },
        underlay last={%
            \draw[#1!40, -{stealth}, line width=1pt] (boxnumber|-frame.north)--(boxnumber|-frame.south);
        },
    }
}

\newtcolorbox[use counter=myboxcounter]{example}[2][]{%
    myboxstyle=#2, title=Example, #1
}

\newtcolorbox[use counter=myboxcounter]{solution}[2][]{%
    myboxstyle=#2, title=Solution, #1
}

\begin{example}{blue}
    hello
\end{example}

\begin{solution}{red}
    hello
\end{solution}

\begin{example}{green}
    hello
\end{example}

\begin{solution}{yellow}
    hello
\end{solution}

\begin{tcolorbox}[myboxstyle=red!75!black]
    This is a \textbf{tcolorbox}.
\end{tcolorbox}



% CODE
\begin{Listing}[t]
    \centering
    % \mbox{
    %     \showrust*[.3\textwidth]{typeinference.rs}
    %     \quad
    %     \showml*[.3\textwidth]{typeinference.ml}
    % }
    \begin{minted}[linenos, breaklines]{rust}
        fn main() {
            let x = 42;
            let y = 3.14;
            let z = x + y;
            println!("The sum is: {}", z);
        }
    \end{minted}
    \caption{Example of type inference in Rust and OCaml}
    \label{lst:typeinference}
\end{Listing}


% TABLES
\begin{table}[t]
    \rowcolors{2}{green!80!yellow!50}{green!70!yellow!40}
    \setlength\arrayrulewidth{1pt}
    \setlength\arrayrulewidth{1pt}
    \centering
    \begin{tabular}{|c|c|c|c|}
        \hline
        \textbf{Language} & \textbf{Type System} & \textbf{Type Checking} & \textbf{Type Inference} \\
        \hline
        C & Weak & Static & No \\
        OCaml & Strong & Static & Yes \\
        Java & Strong & Static  & Yes \\
        Rust & Strong & Static  & Yes \\
        Python & Strong & Dynamic  & Yes \\
        JavaScript & Weak & Dynamic  & Yes \\
        Haskell & Strong & Static  & Yes \\
        Erlang & Strong & Dynamic  & Yes \\
        Perl & Weak & Dynamic  & No \\
        \hline
    \end{tabular}
    \caption{Examples of programming languages and their type systems}
    \label{tab:TypeSystems}
\end{table}

\begin{table}[t]
    \rowcolors{2}{gray!25}{white}
    % \setlength\arrayrulewidth{0pt}
    \centering
    \begin{tabular}{ c c c c }
        \\\toprule \textbf{Language} & \textbf{Type System} & \textbf{Type Checking} & \textbf{Type Inference} \\
        \midrule
        C & Weak & Static & No \\
        OCaml & Strong & Static & Yes \\
        Java & Strong & Static  & Yes \\
        Rust & Strong & Static  & Yes \\
        Python & Strong & Dynamic  & Yes \\
        JavaScript & Weak & Dynamic  & Yes \\
        Haskell & Strong & Static  & Yes \\
        Erlang & Strong & Dynamic  & Yes \\
        Perl & Weak & Dynamic  & No \\
        \bottomrule
    \end{tabular}
    \caption{Examples of programming languages and their type systems}
    \label{tab:TypeSystems}
\end{table}


