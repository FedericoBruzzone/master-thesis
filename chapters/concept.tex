\chapter{Concept}\label{chap:Concept}

\bfcomment{Talk about the importance of modularity}

\section{The type system}\label{sec:concept:TypeSystem}

In \ref{subsec:background:TypeSystems}, we introduce the concept of type systems and we give a brief overview of \textit{type checking} and \textit{type inference}. Our goal is to have solid \textit{application programming interfaces} (APIs) to build type systems for each language.
In this section, we will discuss the importance of the type system for our concept.
What we are looking for in our type system is:
\begin{itemize}
    \item \textbf{modularization}, that allows the definition of custom types and operations on these types, and the ability to combine them in a modular way;
    \item \textbf{flexibility}, since the type system is not known \textit{a priori}, we need the ability to adapt the type system to the specific needs;
    \item \textbf{easy-to-use}, extending the default implementation of the type system with a new type, or implementing a new type from scratch should be easy and straightforward.
\end{itemize}

It is trivial to remember that the type system should also provide the basic functionalities of a type system, such as \textbf{type inference}, that allows the compiler to infer the type of an expression without the need to explicitly specify it; and \textbf{type checking}, that allows the compiler to check if the types of the expressions are correct.

\subsection{The relevance of the type system in the LSP design}\label{subsec:concept:RelevanceOfTheTypeSystem}

The type system is the core of the LSP design. We illustrate the need of having a type system by focusing on the ability to respond to requests from a \textit{Language Client}.

In the reminder of this section, we will evaulate the relevance for three of the most important LSP feautre introduced in \ref{subsec:bacground:KeyMethodsOverview}.





\subsection{Towards a modular type system}\label{sec:concept:TowardsAModularTypeSystem}

\subsection{Type inference}\label{sec:concept:type-inference}

\section{Toward a number of combinations equal to $\mathcal{O}(\ell)$ from $\mathcal{O}(\ell + \mathit{e})$}


