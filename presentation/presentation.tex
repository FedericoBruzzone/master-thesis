\documentclass[9pt,xcolor=table,svgnames]{beamer}
\usepackage{showcode}
% \usetheme{adapt-lab}
\usetheme{customadapt-lab}


\author{Federico Cristiano Bruzzone}
\title[]{Toward a Modular Approach for Type Systems and LSP generation}
\date{July 15\textsuperscript{th} 2024}
\institute{Università degli Studi di Milano \\
           Facoltà di Scienze e Tecnologie \\
           Corso di Laurea Magistrale in Informatica \\ [5pt]\vspace{15pt}
           {\normalsize Advisor: Prof. Walter Cazzola} \\ [1pt]\vspace{5pt}
           {\normalsize Co-Advisor: Dr. Luca Favalli}}

\tolerance=10
\emergencystretch=\maxdimen
\hyphenpenalty=10000
\hbadness=10000

\begin{document}

% \setbeamertemplate{title page}[default][colsep=-4bp,rounded=true,shadow=\beamer@themerounded@shadow]
% \usebackgroundtemplate{
%   \begin{tikzpicture}
%     \path [outer color = white, inner color = blue!1]
%       (0,0) rectangle (\paperwidth,\paperheight);
%   \end{tikzpicture}
% }




% \title[Micro-Languages Variability]{Language Product Lines\\and the Variability of Micro-Languages}
% \date{January 9\textsuperscript{th} 2023}
% \institute{Università degli Studi di Milano\\
%            Facoltà di Scienze e Tecnologie\\\vspace{5pt}
% 		   Bando ID: 5506 - Prof. Cazzola}

\begin{frame}
	\titlepage
\end{frame}

\section[DSL]{Domain Specific Languages (DSLs)}
\subsection{Introduction}
\begin{frame}{\secname}
	\framesubtitle{Introduction}
	A {\color{BloodRed} DSL} is a programming language that mimics the terms, idioms and expressions used among the experts in the target domain
	\medskip

	\begin{itemize}
		\item problem-tailored solutions
		\begin{itemize}
			\item i.e., solutions more concise and clear
		\end{itemize}
		\item domain-oriented solutions
		\begin{itemize}
			\item ideally, a domain expert, with no experience in programming, can read, understand and validate such code
		\end{itemize}
	\end{itemize}
	\bigskip\pause

	The final aim is to increase \alert{productivity} and \alert{software quality} while reducing development costs and \alert{time-to-market}
	\bigskip

	Especially in \alert{pervasive} systems, such as IoT applications
\end{frame}


\section{Problem}
\subsection{Test}

\begin{frame}{\secname}
\framesubtitle{\subsecname}
    Management of {\color{BloodRed} variability} in language product lines is an open problem in software engineering research.\bigskip

    Language variants can be used to customize a base language.
    \begin{itemize}
        %\item Language {\color{BloodRed} features} description.
        %\item Leveraging {\color{BloodRed} similarities}.
        %\item Improve {\color{BloodRed} reuse}.
        \item Switching language syntax and/or semantics.
            \begin{itemize}
                \item Localization.
                \item Sequential vs parallel execution.
                \item Local vs Client-Server.
            \end{itemize}
        \item Support for additional data types.
        \item Enforcing input sanitization.
        \item Additional constructs from other languages.
    \end{itemize}
\end{frame}

\end{document}

